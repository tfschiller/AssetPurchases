\documentclass{article}
\title{The Effect of Asset Purchase Programmes on Asset Prices and Inequality}
\author{Thomas Schiller}
\usepackage[margin=1.0in]{geometry}
\begin{document}
\maketitle{}
\section{Introduction}

\section{Literature Review}
\begin{itemize}
  \item Explanation of the theory behind LSAP (Gagnon et al. 2010)
  \item Transmission mechanisms following Joyce, Lasaosa, Stevens and Tong (2011):
  \begin{enumerate}
    \item macro/policy news
    \item portfolio balance
    \item liquidity premia
  \end{enumerate}
\end{itemize}

\section{Data}
\subsection{Asset Prices}
Distinguishing between different types of assets purchased by the Fed?\\
\\
Timespan: 01-01-2004 - 01-09-2019\\
\\
\begin{itemize}
  \item Asset purchases: Securities Held Outright: U.S. Treasury Securities: All: Wednesday Level
  \item Asset prices: nominal and inflation-indexed bonds, stocks and U.S. dollar exchange rate (vs Euro, Canadian dollar, British pound, Swiss franc, Japanese Yen)
  \item Real Estate: S\&P/Case-Shiller U.S. National Home Price Index
  \item Wilshire 5000 Total Market Full Cap Index
  \item https://fred.stlouisfed.org/series/T5YIFR
  \item https://fred.stlouisfed.org/series/CSUSHPINSA
  \item https://fred.stlouisfed.org/series/TREAST

\end{itemize}
\subsection{Inequality}


\section{Empirics}
\subsection{Methodology}
VAR lag order set to two based on Hesse, Hofmann \& Weber (2018) who in turn follow Weale and Wieladek (2016). \\
Longer term effect - ie not focussed on asset purchase shock announcement ('event study') to ensure consistency with longer term analysis of effect on inequlity.
\subsection{Asset Prices}
\subsection{Inequality}

\section{Results}
\section{Conclusions}
\begin{itemize}
  \item Policy implication
\end{itemize}
\section{References}
\begin{itemize}
  \item Joyce, M., Lasaosa, A., Stevens, I., \& Tong, M. (2011). The financial market impact of quantitative easing in the United Kingdom. International Journal of Central Banking, 7(3), pp. 113-161.
\end{itemize}

\end{document}
